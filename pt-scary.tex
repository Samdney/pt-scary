% ================================================================================
\documentclass[sigconf, screen]{acmart}

\usepackage{booktabs} % For formal tables
\usepackage{graphicx}
\usepackage{comment}
\usepackage{url}
\usepackage{hyperref}

% Copyright
\setcopyright{none}
%\setcopyright{acmcopyright}
%\setcopyright{acmlicensed}
%\setcopyright{rightsretained}
%\setcopyright{usgov}
%\setcopyright{usgovmixed}
%\setcopyright{cagov}
%\setcopyright{cagovmixed}

% ================================================================================
% DOI
%\acmDOI{10.475/123_4}
% ================================================================================
%Conference
%\acmConference[WOODSTOCK'97]{ACM Woodstock conference}{July 1997}{El Paso, Texas USA}
%\acmYear{1997}
%\copyrightyear{2016}

%\acmArticle{4}
%\acmPrice{15.00}
% ================================================================================
% These commands are optional
%\acmBooktitle{Transactions of the ACM Woodstock conference}
%\editor{ABC}
% ================================================================================
\begin{document}
% ================================================================================
\title{Scary: A really scary Pluggable Transport}
% ================================================================================
%\titlenote{Produces the permission block, and copyright information}
\subtitle{My subtitle}	% TODO
\subtitlenote{The author believes in the importance of the independence of research and is funded by the public community. If you also believe in this values, you can find ways for supporting the author's work here: \url{https://research.carolin-zoebelein.de/crowdfunding.html}}
% ================================================================================
\author{Carolin Z\"obelein}
\authornote{\url{https://research.carolin-zoebelein.de}, \textit{E-mail address:} contact@carolin-zoebelein.de, PGP: D4A7 35E8 D47F 801F 2CF6 2BA7 927A FD3C DE47 E13B}
\affiliation[obeypunctuation=true]{
	\institution{Independent mathematical scientist}\\
	\streetaddress{Josephsplatz 8},
	\postcode{90403}
  	\city{N\"urnberg},
  	\country{Germany}  	
}
% ================================================================================
\begin{abstract}	% TODO
STATUS: Draft
%This paper provides a sample of a \LaTeX\ document which conforms,
%somewhat loosely, to the formatting guidelines for
%ACM SIG Proceedings.\footnote{This is an abstract footnote}
\end{abstract}
% ================================================================================
%
% The code below should be generated by the tool at
% http://dl.acm.org/ccs.cfm
% Please copy and paste the code instead of the example below.
%
\begin{comment}	% TODO
\begin{CCSXML}
<ccs2012>
 <concept>
  <concept_id>10010520.10010553.10010562</concept_id>
  <concept_desc>Computer systems organization~Embedded systems</concept_desc>
  <concept_significance>500</concept_significance>
 </concept>
 <concept>
  <concept_id>10010520.10010575.10010755</concept_id>
  <concept_desc>Computer systems organization~Redundancy</concept_desc>
  <concept_significance>300</concept_significance>
 </concept>
 <concept>
  <concept_id>10010520.10010553.10010554</concept_id>
  <concept_desc>Computer systems organization~Robotics</concept_desc>
  <concept_significance>100</concept_significance>
 </concept>
 <concept>
  <concept_id>10003033.10003083.10003095</concept_id>
  <concept_desc>Networks~Network reliability</concept_desc>
  <concept_significance>100</concept_significance>
 </concept>
</ccs2012>
\end{CCSXML}

\ccsdesc[500]{Computer systems organization~Embedded systems}
\ccsdesc[300]{Computer systems organization~Redundancy}
\ccsdesc{Computer systems organization~Robotics}
\ccsdesc[100]{Networks~Network reliability}
\end{comment}

\keywords{Tor, Bridge, Scary, Obscuration, Censorship, Circumvention, Pluggable Transport}	% TODO
% ================================================================================
\maketitle
% ================================================================================
\begin{comment}
TODO: Roadmap
=> Introduction
    - Why do we need Pluggable Transports?
	- History
    - Why do we need more/new/other Pluggable Transports? (Problem of censorships (current state/situation))
=> Definitions (Bridge, Pluggable Transport), Notations, Organization/Outline
=> Explaining the most important Pluggable Transports
-------------------
What we learned from the already existing pluggable transports (advantages/disadvantages, what is good/what is bad)?
Main constraints/consequences which follow from this for the design of a new Pluggable Transport!!!
Looking for additional/already done reasearch/paper work. -> Repetition/Review
A sketch of the basic structure for the PT which follows from this.
Going into details for each part (maybe for each one an own section(?))
-------------------
Ideas for "penetration testing" of this PT
Where could it have weaknesses?
Possible ways for an attack
-------------------
Prospect and conclusions for this PT and the design of future ones.
-------------------
\end{comment}
% ================================================================================
\section*{Preamble}
\label{s:preamble}
% --------------------------------------------------------------------------------
This paper was written in the context of a job application as Pluggable Transport Software Developer for Anti-Censorship Team of The Tor Project\footnote{The Tor Project, Inc., is a 501(c)(3) nonprofit organization advancing human rights and freedoms by creating and deploying free and open source anonymity and privacy technologies. \cite{JobDes}}.
% --------------------------------------------------------------------------------
\section{Introduction}
\label{s:introduction}
% --------------------------------------------------------------------------------
In August 2018, The Intercept published a story about plans of Google for launching a censored version of its search engine in China, which will blacklist websites and search terms about human rights, democracy, religion, and peaceful protest \cite{Dragonfly}. This project, with the code-name \textit{Dragonfly}, started in spring prior year, is the newest step in the ongoing work of creating a censored environment of information in China.

If we look back, the story of censorship in China started in 1998. The Communist Party feared that the China Democracy Party would create a powerful new network. The China Democracy Party was immediately banned, members arrested and imprisonment \cite{GreatFirewallWikiEn}. Finally, this resulted in the beginning of the \textit{Great Firewall (GFW)} project, a combination of legislative actions and technologies enforced by the People's Republic of China to regulate the Internet domestically. It blocks access to selected websites, internet tools, mobile apps and slows down cross-border internet traffic.

Since the GFW blocks destinations and inspects the data being transmitted, ways for censorship circumvention need proxy nodes and encrypted data traffic. Typically, this is done these days by the help of foreign proxy servers, regional website mirrors, Tor, virtual private networks (VPNs) and secure shell (SSH).

Over the years, more and more of this circumvention tools have been blocked due to deep packet inspections and the detailed analysis of its content. So now, many VPNs are no longer useable to circumvent the Great Firewall of China and also the access to the Tor anonymity network \cite{Tor}, with its public list of relays, is no longer possible. 

To solve the problem of relay blocking, Tor introduced so-called \textit{bridges} \cite{TorBridges} which are non-public relays, to help censored users reach the Tor network. Because of the ability of dynamically blocking bridges by looking for their TLS fingerprint \cite{foci12-winter} \cite{Ensafi2015AnalyzingTG}, packet fragmentation and Tor obfsproxy in combination with private bridges, were added \cite{foci12-winter}.

Finally, this lead us to \textit{Pluggable Transports (PT)} \cite{TorPluggableTransports}, which help to bypass censorship attempts against Tor. PTs transform the Tor traffic between client and bridges, in such a way that it looks like innocent traffic instead of the actual Tor traffic. In this paper, we will talk about this PTs, their general construction constraints and an introducing of an sketch of a new PT called \textit{Scary}.
% ---------------------------------------
\subsection{Outline}
\label{ss:outline}
% ---------------------------------------
% TODO
% ---------------------------------------
\subsection{Notation}
\label{ss:notation}
% ---------------------------------------
% TODO
% ================================================================================
\section{TLS fingerprinting}
\label{s:tlsfingerprinting}
% --------------------------------------------------------------------------------




% ================================================================================
%\section{State of the Art}
%\label{s:stateoftheart}
% --------------------------------------------------------------------------------
%In this section we want to make a short review about the most important, already existing, Pluggable Transports.
% ================================================================================
\section{Conclusions}
\label{s:conclusions}
% --------------------------------------------------------------------------------
% TODO
% ================================================================================
\appendix
\section{Appendix}
\label{s:appendix}
% --------------------------------------------------------------------------------

% ---------------------------------------
\subsection{Definitions}
\label{ss:definitions}
% ---------------------------------------
\begin{itemize}
	\item \textbf{Virtual private network (VPN).} TODO		%TODO
	\item \textbf{Secure shell (SSH).} TODO 				%TODO
	\item \textbf{Bridges.}	TODO 							%TODO
	\item \textbf{Pluggable Transport (PT).} TODO 			%TODO
\end{itemize}
% ================================================================================
%\nocite{*}		% TODO
\bibliography{pt-scary}
\bibliographystyle{ACM-Reference-Format}
% ================================================================================
\section*{License}
\label{s:license}
% --------------------------------------------------------------------------------
\begin{center}
	\includegraphics{by-nc-nd.png} \\
	\url{https://creativecommons.org/licenses/by-nc-nd/4.0/}
\end{center}
% ================================================================================
\end{document}
% ================================================================================
