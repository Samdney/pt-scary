% ================================================================================
\documentclass[sigconf, screen]{acmart}

\usepackage{booktabs} % For formal tables
\usepackage{graphicx}
\usepackage{comment}
\usepackage{url}
\usepackage{hyperref}

% Copyright
\setcopyright{none}
%\setcopyright{acmcopyright}
%\setcopyright{acmlicensed}
%\setcopyright{rightsretained}
%\setcopyright{usgov}
%\setcopyright{usgovmixed}
%\setcopyright{cagov}
%\setcopyright{cagovmixed}

% ================================================================================
% DOI
%\acmDOI{10.475/123_4}
% ================================================================================
%Conference
%\acmConference[WOODSTOCK'97]{ACM Woodstock conference}{July 1997}{El Paso, Texas USA}
%\acmYear{1997}
%\copyrightyear{2016}

%\acmArticle{4}
%\acmPrice{15.00}
% ================================================================================
% These commands are optional
%\acmBooktitle{Transactions of the ACM Woodstock conference}
%\editor{ABC}
% ================================================================================
\begin{document}
% ================================================================================
\title{Scary: A really scary Pluggable Transport}
% ================================================================================
%\titlenote{Produces the permission block, and copyright information}
\subtitle{My subtitle}	% TODO
\subtitlenote{The author believes in the importance of the independence of research and is funded by the public community. If you also believe in this values, you can find ways for supporting the author's work here: \url{https://research.carolin-zoebelein.de/crowdfunding.html}}
% ================================================================================
\author{Carolin Z\"obelein}
\authornote{\url{https://research.carolin-zoebelein.de}, \textit{E-mail address:} contact@carolin-zoebelein.de, PGP: D4A7 35E8 D47F 801F 2CF6 2BA7 927A FD3C DE47 E13B}
\affiliation[obeypunctuation=true]{
	\institution{Independent mathematical scientist}\\
	\streetaddress{Josephsplatz 8},
	\postcode{90403}
  	\city{N\"urnberg},
  	\country{Germany}  	
}
% ================================================================================
\begin{abstract}	% TODO
STATUS: Draft
%This paper provides a sample of a \LaTeX\ document which conforms,
%somewhat loosely, to the formatting guidelines for
%ACM SIG Proceedings.\footnote{This is an abstract footnote}
\end{abstract}
% ================================================================================
%
% The code below should be generated by the tool at
% http://dl.acm.org/ccs.cfm
% Please copy and paste the code instead of the example below.
%
\begin{comment}	% TODO
\begin{CCSXML}
<ccs2012>
 <concept>
  <concept_id>10010520.10010553.10010562</concept_id>
  <concept_desc>Computer systems organization~Embedded systems</concept_desc>
  <concept_significance>500</concept_significance>
 </concept>
 <concept>
  <concept_id>10010520.10010575.10010755</concept_id>
  <concept_desc>Computer systems organization~Redundancy</concept_desc>
  <concept_significance>300</concept_significance>
 </concept>
 <concept>
  <concept_id>10010520.10010553.10010554</concept_id>
  <concept_desc>Computer systems organization~Robotics</concept_desc>
  <concept_significance>100</concept_significance>
 </concept>
 <concept>
  <concept_id>10003033.10003083.10003095</concept_id>
  <concept_desc>Networks~Network reliability</concept_desc>
  <concept_significance>100</concept_significance>
 </concept>
</ccs2012>
\end{CCSXML}

\ccsdesc[500]{Computer systems organization~Embedded systems}
\ccsdesc[300]{Computer systems organization~Redundancy}
\ccsdesc{Computer systems organization~Robotics}
\ccsdesc[100]{Networks~Network reliability}
\end{comment}

%\keywords{ACM proceedings, \LaTeX, text tagging}	% TODO
% ================================================================================
\maketitle
% ================================================================================
\begin{comment}
TODO: Roadmap
=> Introduction
=> State of the Art

-------------------
Definitions (Bridge, Pluggable Transport)
History
Explaining the most important Pluggable Transports
Why do we need Pluggable Transports?
Why do we need more/new/other Pluggable Transports? (Problem of censorships (current state/situation))
-------------------
What we learned from the already existing pluggable transports (advantages/disadvantages, what is good/what is bad)?
Main constraints/consequences which follow from this for the design of a new Pluggable Transport!!!
Looking for additional/already done reasearch/paper work. -> Repetition/Review
A sketch of the basic structure for the PT which follows from this.
Going into details for each part (maybe for each one an own section(?)
-------------------
Ideas for "penetration testing" of this PT
Where could it have weaknesses?
Possible ways for an attack
-------------------
Prospect and conclusions for this PT and the design of future ones.
-------------------
\end{comment}
% ================================================================================
\section*{Preamble}
\label{s:preamble}
% --------------------------------------------------------------------------------
This paper was written in the context of a job application as Pluggable Transport Software Developer for Anti-Censorship Team of The Tor Project\footnote{The Tor Project, Inc., is a 501(c)(3) nonprofit organization advancing human rights and freedoms by creating and deploying free and open source anonymity and privacy technologies. \cite{JobDes}}.
% --------------------------------------------------------------------------------
\section{Introduction}
\label{s:introduction}
% --------------------------------------------------------------------------------




% --------------------------------------------------------------------------------
%\section{State of the Art}
%\label{s:stateoftheart}
% --------------------------------------------------------------------------------
%In this section we want to make a short review about the most important, already existing, Pluggable Transports. 






% ================================================================================
\section{Conclusions}	% TODO: conclusions <=> conclusion
\label{s:conclusions}
% --------------------------------------------------------------------------------


% ================================================================================
%\nocite{*}
\bibliography{pt-scary}
\bibliographystyle{ACM-Reference-Format}
% ================================================================================
\section*{License}
\label{s:license}
% --------------------------------------------------------------------------------
\begin{center}
	\includegraphics{by-nc-nd.png} \\
	\url{https://creativecommons.org/licenses/by-nc-nd/4.0/}
\end{center}
% ================================================================================
\end{document}
% ================================================================================
